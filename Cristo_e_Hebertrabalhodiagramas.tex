%! TEX TS-program = lualatex
\documentclass[12pt,openright,oneside,a4paper,
	chapter=TITLE,
	section=TITLE,
	english,brazil]{abntex2}
\usepackage[alf]{abntex2cite}
\usepackage[T1]{fontenc}
\usepackage{fontspec}
\usepackage{graphicx}
\usepackage[none]{hyphenat}
\usepackage{hyperref}
\usepackage{indentfirst}

\setmainfont{Arial}
\setlength{\parindent}{1.5cm}
\setlength{\emergencystretch}{2cm}
\setlength{\cftbeforeparagraphskip}{0pt}
\setlength{\cftbeforesubsubsubsectionskip}{0pt}
\setlength{\cftbeforechapterskip}{0pt}
\setlength{\cftbeforesubsubsectionskip}{0pt}

\author{Heber Ferreira Barra \\ João Gabriel de Cristo}
\titulo{Trabalho de Pesquisa UML}
\date{2024}
\instituicao{Instituto Federal do Paraná}
\local{Curitiba}
\preambulo{Trabalho de pesquisa sobre diagramas UML.}
\tipotrabalho{pesquisa}
\orientador{Elaini Simoni Angelotti}

\makeatletter
\hypersetup{
    pdfauthor={\@author},
    pdfsubject={\imprimirpreambulo},
    pdfcreator={LaTeX with abntex2},
    colorlinks=true,
    linkcolor=black,
    urlcolor=blue,
		citecolor=black
}
\makeatother

\renewcommand{\ABNTEXchapterfont}{\bfseries\MakeUppercase\sffamily}
\renewcommand{\ABNTEXchapterfontsize}{\normal}
\renewcommand{\ABNTEXsectionfont}{\MakeUppercase\sffamily}
\renewcommand{\ABNTEXsectionfontsize}{\normal}
\renewcommand{\labelenumi}{\theenumi}
\renewcommand{\theenumi}{\Roman{enumi} -}%
\renewcommand{\authorcapstyle}{\normal}
\renewcommand{\cftsectionfont}{\normalfont\MakeUppercase}
\newcommand{\bibtextitlecommand}[2]{\bf{#2}}
\newcommand{\authorstyle}{\relax}

\renewcommand{\imprimircapa}{%
	\begin{capa}
		\center
		\ABNTEXchapterfont\bfseries\MakeUppercase\imprimirinstituicao\\
		\vspace*{1cm}
		\ABNTEXchapterfont\bfseries\MakeUppercase\imprimirautor
		\vfill
		\begin{center}
			\ABNTEXchapterfont\bfseries\MakeUppercase\imprimirtitulo
		\end{center}
		\vfill

		\bfseries\MakeUppercase\imprimirlocal

		\bfseries\MakeUppercase\imprimirdata

		\vspace*{1cm}
	\end{capa}
}

\makeatletter
\renewcommand{\folhaderostocontent} {
	\center
	{\normal\imprimirautor}

	\vspace*{\fill}
	\begin{center}
		\ABNTEXchapterfont\bfseries\MakeUppercase\imprimirtitulo
	\end{center}

	\abntex@ifnotempty{\imprimirpreambulo}{%
		\hspace{.45\textwidth}
		\begin{minipage}{.5\textwidth}
			\SingleSpacing
			\small\imprimirpreambulo \\
			\\
			\imprimirorientadorRotulo\ \imprimirorientador
		\end{minipage}
	}

	\vspace*{\fill}
	{\imprimirlocal}
	\par
	\imprimirdata
	\vspace*{1cm}
}
\makeatother

\begin{document}
\selectlanguage{brazil}
\imprimircapa
\imprimirfolhaderosto

\pdfbookmark[0]{\listfigurename}{lof}
\listoffigures*
\cleardoublepage

\pdfbookmark[0]{\contentsname}{toc}
\tableofcontents*
\cleardoublepage

\textual
\chapter{Introdução à UML}

\chapter{Diagrama de Atividades}

\section{Objetivo do Diagrama de Atividades}

\section{Principais Componentes do Diagrama de Atividades}

\section{Exemplo do Diagrama de Atividades}

\chapter{Diagrama de Sequência}

\section{Objetivo do Diagrama de Sequência}

\section{Principais Componentes do Diagrama de Sequência}

\section{Exemplo do Diagrama de Sequência}

\chapter{Diagrama de Comunicação}

\section{Objetivo do Diagrama de Comunicação}

\section{Principais Componentes do Diagrama de Comunicação}

\section{Exemplo do Diagrama de Comunicação}

\chapter{Diagrama de Estrutura Composta}

\section{Objetivo do Diagrama de Estrutura Composta}

\section{Principais Componentes do Diagrama de Estrutura Composta}

\section{Exemplo do Diagrama de Estrutura Composta}

\chapter{Diagrama de Pacotes}

\section{Objetivo do Diagrama de Pacotes}

\section{Principais Componentes do Diagrama de Pacotes}

\section{Exemplo do Diagrama de Pacotes}

\chapter{Objetivos dos demais Diagramas da UML}

\section{Diagrama de Tempo}
\cite{GUEDES}
\cite{BEZERRA}
\cite{LARMAN}

\chapter{Conclusão}

\citeoption{opcoes}
\postextual
\bibliography{referencias}

\end{document}
